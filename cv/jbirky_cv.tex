
\documentclass[a4,11pt]{article}

\usepackage{latexsym}
\usepackage[empty]{fullpage}
% \usepackage[letterpaper,margin=1in]{geometry}
\usepackage{titlesec}
 \usepackage{marvosym}
\usepackage[usenames,dvipsnames]{color}
\usepackage{verbatim}
\usepackage[hidelinks]{hyperref}
\usepackage{fancyhdr}
\usepackage{multicol}
\usepackage{multirow}
\usepackage{array}
\usepackage{hyperref}
\usepackage{csquotes}
\usepackage{tabularx}
\usepackage[11pt]{moresize}
\usepackage{setspace}
\usepackage{fontspec}
\usepackage[inline]{enumitem}
\usepackage{array}
\newcolumntype{P}[1]{>{\centering\arraybackslash}p{#1}}
\usepackage{anyfontsize}
\usepackage{fontawesome}
\usepackage[official]{eurosym}
\usepackage{xcolor}

%%%% Set Secondary Color
\definecolor{UI_blue}{RGB}{32, 64, 151}
\hypersetup{colorlinks=true,urlcolor=UI_blue}

%%%% Set Margins
\usepackage[margin=1cm, top=1cm]{geometry}

% \setlist{nolistsep}

\newcommand{\nlist}[1]{{\color{bcolor} [#1\,]}}

%%%% Set Fonts
\setmainfont[
BoldFont=SourceSansPro-Semibold.otf, %SourceSansPro-Bold.otf
ItalicFont=SourceSansPro-RegularIt.otf
]{SourceSansPro-Regular.otf}
\setsansfont{SourceSansPro-Semibold.otf}

%%%% Set Page
\pagestyle{fancy}
\fancyhf{} 
\fancyfoot{}
\renewcommand{\headrulewidth}{0pt}
\renewcommand{\footrulewidth}{0pt}

% remove footnote line
\renewcommand*\footnoterule{}
% no number footnote
\makeatletter
\def\blfootnote{\xdef\@thefnmark{}\@footnotetext}
\makeatother

%%%% Set URL Style
\urlstyle{same}

%%%% Set Indentation
\raggedbottom
\raggedright
\setlength{\tabcolsep}{0in}

%%%% Define New Commands
\usepackage[style=nature, maxbibnames=3]{biblatex}
\addbibresource{Publications.bib}

%%%% Bold Name in Publications
\renewcommand*{\mkbibnamegiven}[1]{%
\ifitemannotation{highlight}
{\textbf{#1}}
{#1}}

\renewcommand*{\mkbibnamefamily}[1]{%
\ifitemannotation{highlight}
{\textbf{#1}}
{#1}}

%%%% Set Sections formatting
\titleformat{\section}{
\color{UI_blue} \scshape \raggedright \large 
}{}{0em}{}[\vspace{-10pt} \hrulefill \vspace{-6pt}]

%%%% Set Subtext Formatting
\newcommand{\subtext}[1]{
#1\par\vspace{-0.2cm}}

% \newcommand{\subtextit}[1]{\vspace{0.15cm}
% \textit{ #1 \vspace{-0.2cm}} }

%%%% Set Item Spacing
\setlist[itemize]{align=parleft,left=0pt..1em}

%%%% New Itemize "Zitemize" Formatting - tighter spacing than itemize
\newenvironment{zitemize}{
\begin{itemize}\itemsep0pt \parskip0pt \parsep1pt}
{\end{itemize}\vspace{-0.5cm}}


%%%% Define Skills Bold Formatting
\newcommand{\hskills}[1]{
\textbf{\bfseries #1} }

%%%% Set Subsection formatting
\titleformat{\subsection}{\vspace{-0.1cm} 
\bfseries \fontsize{11pt}{2cm}}{}{0em}{}[\vspace{-0.2cm}]

% Redefinition, symbol included in link:
\let\orighref\href
\renewcommand{\href}[2]{\orighref{#1}{#2\,\scriptsize\faExternalLink}}

\newcommand{\code}[2]{\href{#1}{\texttt{#2}}} 

\begin{document}

%%%%%%% --------------------------------------------------------------------------------------
%%%%%%%  HEADER
%%%%%%% --------------------------------------------------------------------------------------
\begin{center}
    \begin{minipage}[b]{0.24\textwidth}
            \flushleft  
            PhD Student \\
            University of Washington \\
            {\href{https://jessicabirky.com}{https://jessicabirky.com} } \\
    \end{minipage}   
    \begin{minipage}[b]{0.5\textwidth}
            \centering
            {\Huge Jessica Birky} \\ %
            \vspace{0.1cm}
    \end{minipage}% 
    \begin{minipage}[b]{0.24\textwidth}
            \flushright
            (510) 364-5254 \\
            \href{mailto:jbirky@uw.edu}{jbirky@uw.edu} \\
            \href{https://orcid.org/0000-0002-7961-6881}{0000-0002-7961-6881}
    \end{minipage}% 

% footnote 
\blfootnote{\centering\normalsize\color{gray} \textit{[Last updated: \today]}}
    
\vspace{-0.15cm} 
{\color{UI_blue} \hrulefill}
\end{center}
\textbf{Research Interests:} Stars, binary stars, stellar populations. Stellar and planetary dynamics. Large scale surveys, data analysis and modeling, machine learning and data-driven models. 
\vspace{-0.2cm}

%%%%%%% --------------------------------------------------------------------------------------
\section{Education }
\textbf{PhD Astronomy} (Data Science Program), University of Washington -- Seattle \hfill 2021 --- Present \\ 
\vspace{0.1cm}
\textbf{MS Astronomy,} University of Washington -- Seattle \hfill 2021 \\
\vspace{0.1cm}
\textbf{BS Physics,} University of California -- San Diego \hfill 2019


%%%%%%% --------------------------------------------------------------------------------------
\section{Research Experience} 

\textbf{Graduate Student Researcher} -- University of Washington, Seattle WA \hfill Aug 2020 -- Present \\
\textsl{VPLanet Group; Advisor: Rory Barnes} \\
% Topic: Inferring tidal evolution of binary stars using \texttt{VPLanet}
\vspace{.2cm} 

\textbf{Graduate Student Researcher} -- University of Washington, Seattle WA \hfill Aug 2019 -- Present \\
\textsl{DIRAC Institute; Advisor: James Davenport} \\
% Topic: Identifying and measuring dynamics of eclipsing binaries in \textsl{TESS}, with radial velocity follow-up using APO 3.5m
\vspace{.2cm} 

\textbf{Undergraduate Researcher} -- University of California San Diego, La Jolla CA \hfill May 2016 -- May 2019  \\
\textsl{Cool Star Lab; Advisor: Adam Burgasser} \\
% Topic: Implemented a forward modelling pipeline for inferring atmospheric and kinematic parameters of low-mass stars  and brown dwarfs from high resolution spectra
\vspace{.2cm} 

\textbf{Research Intern} -- Max Planck Institute f{\"u}r Astronomie, Heidelberg DE \hfill Summer 2017 \& 2018 \\
\textsl{Stars \& Milky Way groups; Advisor: David Hogg} \\
% Topic: Trained machine learning models of M dwarf spectra to precisely predict temperatures and metallicities of M dwarfs


%%%%%%% --------------------------------------------------------------------------------------
\section{Selected Awards \& Honors}

\begin{tabular}[t]{ p{9em} p{50em}}
\hskills{2024} & UW Astronomy Excellence in Teaching Award \\
\hskills{2024} & Washington Space Grant Fellowship \\
\hskills{2019 -- 2024} & \textbf{NSF Graduate Research Fellowship} \\ %(\hskills{\$144k}/3 yrs ) \\ % 
\hskills{2022, \, 2024} & UW Astronomy Jacobsen Award  \\
\hskills{2018, \, 2019} & UC San Diego Provost Honors  \\ % 
\hskills{2017, \, 2018} & Max Planck Institute Fellowship \\ %(\hskills{\EUR6k}/2 summers ) \\
\hskills{2017} & Frances Hellman Research Scholarship \\  %(\hskills{\$5k}) \\
\hskills{2016 -- 2018} & Physics Chair Challenge Award \\ %(\hskills{\$900}) \\
\end{tabular} 


%%%%%%% --------------------------------------------------------------------------------------
\section{Publications} 

% \href{https://ui.adsabs.harvard.edu/search/fq=%7B!type%3Daqp%20v%3D%24fq_database%7D&fq_database=(database%3Aastronomy%20OR%20database%3Aphysics)&p_=0&q=%20author%3A%22birky%2C%20jessica%22&sort=date%20desc%2C%20bibcode%20desc}{ADS}

% \href{https://scholar.google.com/citations?user=rq4yu_UAAAAJ&hl=en}{Google Scholar}

\textsl{Lead-author Papers (underlined = undergraduates mentored):} 
\begin{itemize}[itemsep=0pt]
    \item \textbf{Birky, J.}, Barnes, R. K., 2023, \textsl{Prospects of Constraining Tidal Dissipation in Binary Stars} (Submitted ApJ)
    \item \textbf{Birky, J.}, Barnes, R. K., Davenport, J. R. A., 2023, \textsl{ALABI: Active Learning for Accelerated Bayesian Inference} (In prep)
    \item \underline{Hobson-Ritz, M.}, \textbf{Birky, J.}, \underline{Peterson, L.}, \underline{Gwartney, P.}, \underline{Delker, J.}, \underline{Wong, R.}, Gordon, T., Davenport, J. R. A., Barnes, R. K., 2023, \textsl{Tidal Synchronization of TESS Eclipsing Binaries} (In Prep)
    \item \textbf{Birky, J.}, Barnes, R. K., Fleming, D. P., 2021, \textsl{Improved Constraints for Trappist-1 XUV Luminosity Evolution}, RNAAS, 5, 122 (arXiv:2105.12562)  [\href{https://iopscience.iop.org/article/10.3847/2515-5172/ac034c}{paper}] [\href{https://github.com/jbirky/trappist_xuv}{code}]
    \item \textbf{Birky, J.}, Hogg, D. W., Mann, A., Burgasser, A. J., 2020, \textsl{Temperatures and Metallicities for M dwarfs in the APOGEE Survey}, ApJ, 892, 1 (arXiv:2001.04962) [\href{https://iopscience.iop.org/article/10.3847/1538-4357/ab7004}{paper}] [\href{https://github.com/jbirky/Mdwarf_project}{code}]
\end{itemize}

\textsl{Co-author Papers:} 
\begin{itemize}[itemsep=0pt]
    % \item Barnes, R., K., \textbf{Birky J.} et. al 2023, \textsl{History and Habitability of the LP 791-18 Planetary System} (In prep)
    \item Barnes, R. K., Amaral L., \textbf{Birky J.}, et al. 2023, \textsl{History and Habitability of the LP 890-9 Planetary System} (Accepted PSJ) 
    \item Gialluca, M. T., Barnes, R. K.,  et. al (incl. \textbf{Birky, J.}) 2024, \textsl{Bayesian Calculations of Water Inventories and Oxygen Accumulation on the TRAPPIST-1 Planets}, PSJ, 5, 137  [\href{https://iopscience.iop.org/article/10.3847/PSJ/ad4454/meta}{paper}]
    \item Hsu, C., Burgasser, A. J., et al. (incl. \textbf{Birky, J.}) 2024, \textsl{The Brown Dwarf Kinematics Project (BDKP). VI: Ultracool Dwarf Radial and Rotational Velocity Survey with SDSS/APOGEE High-Resolution Spectrometer}, \textit{Accepted ApJS} (arXiv:2403.13760) [\href{https://arxiv.org/abs/2403.13760}{paper}]
    \item Hsu, C., Burgasser, A. J., et al. (incl. \textbf{Birky, J.}) 2021, \textsl{The Brown Dwarf Kinematics Project (BDKP). V. Radial and Rotational Velocities of T Dwarfs From Keck/NIRSPEC High-Resolution Spectroscopy} (arXiv:2107.01222)  [\href{https://arxiv.org/abs/2107.01222}{paper}] [\href{https://github.com/chihchunhsu/smart}{code}] 
    \item Davenport, J. R. A., Windemuth, D., et al. (incl. \textbf{Birky, J.}) 2021, \textsl{The Rise and Fall of the Eclipsing Binary, HS Hydra}, \textit{ApJL} (arXiv:2107.10954) [\href{https://arxiv.org/abs/2107.10954}{paper}] 
    \item Martin, D. V., El-Badry, K., et al. (incl. \textbf{Birky, J.}) 2021, \textsl{TOI-1259Ab--a gas giant with 2.6\% deep transits and a bound white dwarf companion}, \textit{MNRAS} (arXiv:2101.02707) [\href{https://arxiv.org/abs/2101.02707}{paper}]
    \item Burgasser, A. J., Splat Development Team (incl. \textbf{Birky, J.}), \textsl{The SpeX Prism Library Analysis Toolkit (SPLAT): A Data Curation Model}, Bull. Astr. Soc. India, 00, 1-6, 2017 (arXiv:1707.00062)
\end{itemize}


%%%%%%% --------------------------------------------------------------------------------------
\section{Talks / Posters}

\textit{Prospects of Constraining Tidal Dissipation of Low-mass Stars} \hfill 2024 \\
Poster presentation at Cool Stars 22, UC San Diego  \vspace{.2cm} \newline
\textit{The Formation of Short-period Binaries in Hierarchical Triples} \hfill 2024 \\
General Exam, University of Washington  \vspace{.2cm} \newline
\textit{Prospects of Constraining Tidal Dissipation of Low-mass Stars} \hfill 2023 \\
Qualifying Exam, University of Washington  \vspace{.2cm} \newline
\textit{Prospects of Constraining Tidal Dissipation of Low-mass Stars} \hfill 2023 \\
Talk at DDA Meeting 54, Lansing, Michigan  \vspace{.2cm} \newline
\textsl{Challenges in Establishing an Accurate Model of Tidal Dissipation for Low-mass Binary Stars.} \hfill 2023 \\ Poster presentation at AAS 241, Seattle WA  \vspace{.2cm} \newline
\textit{\textbf{(Invited)} Precise abundances of M dwarfs: data driven models applied to large scale surveys} \hfill 2022 \\
Talk at Cool Stars 21, Toulouse, France  \vspace{.2cm} \newline
\textsl{Constraining the XUV Luminosity Evolution of Low Mass Stars.} \hfill 2022 \\
Poster presentation at Cool Stars 21, Toulouse France \vspace{.2cm} \newline
\textit{ALABI: Active Learning for Accelerated Bayesian Inference} \hfill 2021 \\
IAU Symposium 362 -- Predictive Power of Computational Astrophysics, Virtual Conference \vspace{.2cm} \newline
\textsl{Systematic Classification of TESS Eclipsing Binaries.} \hfill 2020 \\
Poster presentation at AAS Meeting 235, Honolulu HI [\href{https://doi.org/10.5281/zenodo.3605647}{poster}] \vspace{.2cm} \newline
\textit{Physical Parameters for 10,000+ M dwarfs in the APOGEE Survey} \hfill 2019 \\ 
Talk at Sloan Digital Sky Survey Collaboration Meeting, Ensenada, Mexico \vspace{.2cm} \newline
\textsl{Precise Stellar Parameters for 10,000+ APOGEE M dwarfs.} \hfill 2019 \\
Poster presentation at AAS Meeting 233, Seattle WA [\href{https://doi.org/10.5281/zenodo.2536586}{poster}] \vspace{.2cm} \newline 
\textit{Data-Driven Spectral Models for APOGEE M Dwarfs}. \hfill 2018 \\
Poster presentation at AAS Meeting 231, Washington DC [\href{http://doi.org/10.5281/zenodo.1146909}{poster}] \vspace{.2cm} \newline
\textsl{Modeling Stellar Parameters for High Resolution Late-M and Early-L Dwarf SDSS/APOGEE Spectra} \hfill 2017 \\
Poster presentation at AAS Meeting 229, Grapevine TX [\href{http://doi.org/10.5281/zenodo.1116625}{poster}]  \vspace{.2cm} \newline
\textit{Data Driven Models for APOGEE M dwarfs} \hfill 2017 \\ 
Talk at Stars Meeting  \& Milky Way Meeting, MPIA, Heidelberg, Germany \vspace{.2cm} \newline
\textsl{Identification of H-band Absorption Lines in High Resolution APOGEE Spectra of the Lowest Mass Stars}. \hfill 2016 \\
Poster presentation at the national SACNAS Conference, Long Beach CA 

\clearpage
%%%%%%% --------------------------------------------------------------------------------------
\section{Telescope Time Awarded}

\textsl{TESS Eclipsing Binaries in Open Clusters Survey} \hfill 2022 -- 2023 \\
PI: \textbf{APO 3.5 meter} -- 16 total half nights with ARCES/KOSMOS spectrographs \vspace{.2cm} 

\textsl{Training the Cannon: Calibrating APOGEE Observations of Ultracool Dwarfs} \hfill 2018 -- 2019 \\ 
Co-I: \textbf{NASA IRTF} -- 6 nights with iShell spectrograph (PI: Adam Burgasser) \vspace{.2cm} 

\textsl{APOGEE-2 Survey of the Lowest-Mass Stars and Brown Dwarfs: Composition, Chemistry and Companions} \hfill 2017 -- 2018 \\
Co-I: \textbf{APOGEE 2.5-meter} -- Fibers for ancillary survey (PI: Adam Burgasser) 

\section{Observing Experience}
\textbf{Apache Point Observatory (APO) 3.5m}  \\
6 half nights (remote), Instruments: ARCES \& KOSMOS \hfill Q1 2023 \\
8 half nights (remote), Instruments: ARCES \& KOSMOS \hfill Q4 2022 \\
4 half nights (onsite), Instruments: ARCES, KOSMOS, ARCTIC \hfill Q3 2022 \\
2 half nights (remote), Instruments: ARCES \& KOSMOS \hfill Q2 2022 \\
2 half nights (remote), Instruments: ARCES, TripleSpec, DIS, NICFPS, ARCTIC \hfill Q4 2020 \\

\vspace{.2cm}
\textbf{Manastash Ridge Observatory (MRO) 30in} \\
1 full night (onsite) \hfill Q4 2023


%%%%%%% --------------------------------------------------------------------------------------
\section{Research Mentorship}

\textbf{Co-founder of YVC-UW Partnership for Research in Astrophysics (YUPRA)} \hfill Summer 2024 \\
\vspace{.1cm}
Co-founded a partnership program between Yakima Valley Community College (YVC) and University of Washington (UW) designed to expose underrepresented students from eastern Washington to research. Received \$50k grant from Washington Space grant to operate the summer 2024 REU and provide student stipends. Designed student research projects and served as primary research mentor for six YVC undergraduates in summer 2024. Project webpage: \href{https://jessicabirky.com/yupra/}{https://jessicabirky.com/yupra/}  \\

\vspace{.3cm}
\textbf{Mentor for Pre–Major in Astronomy Program (Pre-MAP)} \hfill Fall 2022 \\
\vspace{.1cm}
Served as mentor for \href{http://depts.washington.edu/premap/about/}{Pre-MAP}: a program designed to expose incoming UW students to programming and/or scientific research who are traditionally underrepresented in astronomy, such as low-income and/or first-generation college students. \\

\vspace{.3cm}
\underline{Students mentored:} \\
\vspace{.1cm}
\textbf{Karime Estrada} (YUPRA REU) \hfill Summer 2024 \\
\vspace{.1cm}
\textbf{Josue Cruz} (YUPRA REU) \hfill Summer 2024 \\
\vspace{.1cm}
\textbf{Alexandra Sanchez} (YUPRA REU) \hfill Summer 2024 \\
\vspace{.1cm}
\textbf{Michelle Marquez} (YUPRA REU) \hfill Summer 2024 \\
\vspace{.1cm}
\textbf{Julizza Gomez} (YUPRA REU) \hfill Summer 2024 \\
\vspace{.1cm}
\textbf{Marshall Hobson-Ritz} (UW post-bac; now \underline{PhD student at University of Maryland}) \hfill Jan 2023 -- Present \\
\vspace{.1cm}
\textbf{John Delker} (UW undergrad, PreMAP program; now \underline{PhD student at Michigan State}) \hfill Nov 2022 -- Dec 2022 \\
\vspace{.1cm}
\textbf{Leah Peterson} (UW undergrad, PreMAP program) \hfill Nov 2022 -- Sept 2023 \\
\vspace{.1cm}
\textbf{Peter Gwartney} (UW undergrad; now \underline{PhD student University of Alabama}) \hfill Jun 2021 -- Sept 2022 \\
\vspace{.1cm}
\textbf{Rachel Wong} (UW undergrad, co-mentored with PhD student Tyler Gordon) \hfill Jun 2021 -- Sept 2022 


%%%%%%% --------------------------------------------------------------------------------------
\section{Leadership}

\textbf{Department Grad Student Representative} \hfill Summer 2024 -- \\
Represent and advocate for graduate student interests at department faculty meetings, manage the assignment of grad jobs, and plan graduate student orientation and community events. Also initiated the development of a grad wiki website for documenting resources.
\vspace{.2cm}

\textbf{Lead Teaching Assistant} \hfill Fall 2024 -- \\ 
Developed a new training program for all new TAs in the department. This includes writing a TA handbook, helping TAs develop discussion materials, providing teaching feedback, as well as assisting in solving any class issues. 
\vspace{.2cm}


%%%%%%% --------------------------------------------------------------------------------------
\section{Teaching Experience}

\textbf{Teaching Assistant:} Lead weekly lab or discussion sections, held office hours, and graded assignments/exams for both intro and advanced astronomy courses: \vspace{.3cm} 

\textbf{ASTR 421: Stellar Theory and Observations} (Instructor: Emily Levesque) \hfill Winter 2025 \\
\textsl{Upper-division major course. Observations and theory of the atmospheres, chemical composition, internal structure, energy sources, and evolutionary history of stars.} \vspace{.2cm}

\textbf{ASTR 101: Introduction to Astronomy} (Instructor: Chris Laws) \hfill Spring 2024  \\
\textsl{Introductory course for non-science majors. Introduction to the universe, with emphasis on conceptual, as contrasted with mathematical, comprehension. Modern theories, observations; ideas concerning nature, evolution of galaxies; quasars, stars, black holes, planets, solar system.} \vspace{.2cm}

\textbf{ASTR 421: Stellar Theory and Observations} (Instructor: Emily Levesque) \hfill Winter 2024 \\
\vspace{.2cm}

\textbf{ASTR 150: The Planets} (Instructor: Toby Smith) \hfill Fall 2023  \\
\textsl{Introductory course for non-science majors. Survey of the planets of the solar system, with emphases on recent space exploration of the planets and on the comparative evolution of the Earth and the other planets.} \vspace{.2cm}

\textbf{ASTR 101: Introduction to Astronomy} (Instructor: Chris Laws) \hfill Summer 2023  \\
\vspace{.2cm}

\textbf{ASTR 150: The Planets} (Instructor: Nicole Kelly) \hfill Spring 2021  \\
\vspace{.2cm}

\textbf{ASTR 150: The Planets} (Instructors: Nicole Kelly, Eric Agol) \hfill Winter 2021  \\
\vspace{.2cm}

\textbf{ASTR 102: Introduction to Astronomy} (Instructor: Scott Anderson) \hfill Fall 2020 \\
\textsl{Emphasis on mathematical and physical comprehension of nature, the sun, stars, galaxies, and cosmology.}


%%%%%%% --------------------------------------------------------------------------------------
\section{Other Service and Outreach}

Astronomy Grad Congress Representative \hfill Fall 2024 -- Present \\
Planetarium Volunteer -- outreach shows at the UW planetarium \hfill Fall 2023 -- Present \\
Astronomy on Tap Seattle -- Organizer \& Livestream Manager \hfill Fall 2022 -- Present \\
\texttt{VPLanet} Workshop III -- SOC and Session Lead \hfill Sept 2023 \\
Yakima Valley CC -- organized overnight field trip for students to observe at MRO \hfill Oct 2023 \\
Apache Point Observatory -- Telescope Allocation Committee \hfill Fall 2020 -- Fall 2022 \\
AAS Foundations of astronomical data science workshop -- volunteer helper \hfill Jan 2023 \\
Yakima Valley CC -- Outreach talk: Research in modern Astronomy \hfill Nov 2022 \\
UW Pre-Map program -- Research Mentor \hfill Fall 2022 \\
\texttt{VPLanet} Workshop II -- SOC and Session Lead \hfill Sept 2022 

%%%%%%% --------------------------------------------------------------------------------------
\section{Engineering Experience}
\textbf{Design Team} -- UCSD Human Powered Submarine Club \hfill Sept 2015 --- May 2017  \\
\textsl{Prototyped 3D submarine hull profiles to minimize fluid drag, while subject to constraints of internal diving/safety equipment. Designed submarine propulsion fin mechanism, CADed Solidworks models, and prototyped using 3D printing. }

%%%%%%% --------------------------------------------------------------------------------------
\section{Professional Development}

NBIA Summer School on Astrophysical Dynamics of Gravitating Systems -- \textit{Copenhagen, Denmark} \hfill Aug 2024 \\
AI-driven discovery in physics and astrophysics -- \textit{Tokyo, Japan} \hfill Jan 2024 \\
Cool Stars 20.5 -- \textit{Virtual Conference} \hfill Mar 2021 \\
NExSS Quantitative Habitability Science Workshop -- \textit{Online workshop} \hfill Dec 2020 \\
online.tess.science -- \textit{Online workshop} \hfill Sep 2020 \\
TESS Ninja 3: Expanding the Science of TESS -- \textit{Sydney, Australia} \hfill Feb 2020 \\
ZTF Collaboration Meeting -- \textit{UW Seattle, WA} \hfill Sept 2019 \\
Caltech FUTURE of Physics Workshop -- \textit{Pasadena, CA} \hfill Nov 2018 \\
M33 HST Survey Meeting -- \textit{Ringberg Castle, Tegernsee, Germany} \hfill Jul 2018 \\
Conference for Undergraduate Women in Physics -- \textit{Cal Poly Pomona, CA}  \hfill Jan 2018 \\
Gaia Sprint -- \textit{Internationales Wissenschaftsforum Heidelberg, Germany}  \hfill Jul 2017 \\
Conference for Undergraduate Women in Physics -- \textit{UC Los Angeles, CA}  \hfill Jan 2017 

% %%%%%%% --------------------------------------------------------------------------------------
% \section{Professional Affiliations}

% American Astronomical Society (AAS) Member \hfill 2016 -- Present \\
% Society for the Advancement of Chicanos and Native Americans in Science \hfill 2016 -- 2019 \\
% Sloan Digital Sky Survey (SDSS) -- Faculty and Student Team (FAST) Member \hfill 2016 -- 2019 

%%%%%%% --------------------------------------------------------------------------------------
\section{Graduate Coursework}

Radiative Processes, Thermo/hydrodynamics, Stellar Structure and Evolution, Explanets, Interstellar \& Intergalactic Medium, Galactic Structure \& Dynamics, Astrostatistics, Machine Learning, Data Visualization

%%%%%%% --------------------------------------------------------------------------------------
\section{References}

\textbf{Prof. Rory Barnes} (UWL) -- PhD advisor \hfill {\tt \href{rkb9@uw.edu}{rkb9@uw.edu}} \\
\textbf{Prof. James Davenport} (UW/DIRAC) -- PhD co-advisor \hfill {\tt \href{jrad@uw.edu}{jrad@uw.edu}} \\
\textbf{Prof. Adam Burgasser} (UCSD) -- undergrad research advisor \hfill {\tt \href{aburgasser@ucsd.edu}{aburgasser@ucsd.edu}}   \\
\textbf{Prof. David Hogg} (NYU/MPIA/Flatiron) -- undergrad research advisor \hfill {\tt \href{david.hogg@nyu.edu}{david.hogg@nyu.edu}}  \\


%%%%%%% ---------------------------- END DOC HERE ---------------------------- %%%%%%% 
\end{document}